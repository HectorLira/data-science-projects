\PassOptionsToPackage{unicode=true}{hyperref} % options for packages loaded elsewhere
\PassOptionsToPackage{hyphens}{url}
%
\documentclass[]{article}
\usepackage{lmodern}
\usepackage{amssymb,amsmath}
\usepackage{ifxetex,ifluatex}
\usepackage{fixltx2e} % provides \textsubscript
\ifnum 0\ifxetex 1\fi\ifluatex 1\fi=0 % if pdftex
  \usepackage[T1]{fontenc}
  \usepackage[utf8]{inputenc}
  \usepackage{textcomp} % provides euro and other symbols
\else % if luatex or xelatex
  \usepackage{unicode-math}
  \defaultfontfeatures{Ligatures=TeX,Scale=MatchLowercase}
\fi
% use upquote if available, for straight quotes in verbatim environments
\IfFileExists{upquote.sty}{\usepackage{upquote}}{}
% use microtype if available
\IfFileExists{microtype.sty}{%
\usepackage[]{microtype}
\UseMicrotypeSet[protrusion]{basicmath} % disable protrusion for tt fonts
}{}
\IfFileExists{parskip.sty}{%
\usepackage{parskip}
}{% else
\setlength{\parindent}{0pt}
\setlength{\parskip}{6pt plus 2pt minus 1pt}
}
\usepackage{hyperref}
\hypersetup{
            pdftitle={Chapter 2 - Exercises},
            pdfauthor={Héctor Lira Talancón},
            pdfborder={0 0 0},
            breaklinks=true}
\urlstyle{same}  % don't use monospace font for urls
\usepackage[margin=1in]{geometry}
\usepackage{color}
\usepackage{fancyvrb}
\newcommand{\VerbBar}{|}
\newcommand{\VERB}{\Verb[commandchars=\\\{\}]}
\DefineVerbatimEnvironment{Highlighting}{Verbatim}{commandchars=\\\{\}}
% Add ',fontsize=\small' for more characters per line
\usepackage{framed}
\definecolor{shadecolor}{RGB}{248,248,248}
\newenvironment{Shaded}{\begin{snugshade}}{\end{snugshade}}
\newcommand{\AlertTok}[1]{\textcolor[rgb]{0.94,0.16,0.16}{#1}}
\newcommand{\AnnotationTok}[1]{\textcolor[rgb]{0.56,0.35,0.01}{\textbf{\textit{#1}}}}
\newcommand{\AttributeTok}[1]{\textcolor[rgb]{0.77,0.63,0.00}{#1}}
\newcommand{\BaseNTok}[1]{\textcolor[rgb]{0.00,0.00,0.81}{#1}}
\newcommand{\BuiltInTok}[1]{#1}
\newcommand{\CharTok}[1]{\textcolor[rgb]{0.31,0.60,0.02}{#1}}
\newcommand{\CommentTok}[1]{\textcolor[rgb]{0.56,0.35,0.01}{\textit{#1}}}
\newcommand{\CommentVarTok}[1]{\textcolor[rgb]{0.56,0.35,0.01}{\textbf{\textit{#1}}}}
\newcommand{\ConstantTok}[1]{\textcolor[rgb]{0.00,0.00,0.00}{#1}}
\newcommand{\ControlFlowTok}[1]{\textcolor[rgb]{0.13,0.29,0.53}{\textbf{#1}}}
\newcommand{\DataTypeTok}[1]{\textcolor[rgb]{0.13,0.29,0.53}{#1}}
\newcommand{\DecValTok}[1]{\textcolor[rgb]{0.00,0.00,0.81}{#1}}
\newcommand{\DocumentationTok}[1]{\textcolor[rgb]{0.56,0.35,0.01}{\textbf{\textit{#1}}}}
\newcommand{\ErrorTok}[1]{\textcolor[rgb]{0.64,0.00,0.00}{\textbf{#1}}}
\newcommand{\ExtensionTok}[1]{#1}
\newcommand{\FloatTok}[1]{\textcolor[rgb]{0.00,0.00,0.81}{#1}}
\newcommand{\FunctionTok}[1]{\textcolor[rgb]{0.00,0.00,0.00}{#1}}
\newcommand{\ImportTok}[1]{#1}
\newcommand{\InformationTok}[1]{\textcolor[rgb]{0.56,0.35,0.01}{\textbf{\textit{#1}}}}
\newcommand{\KeywordTok}[1]{\textcolor[rgb]{0.13,0.29,0.53}{\textbf{#1}}}
\newcommand{\NormalTok}[1]{#1}
\newcommand{\OperatorTok}[1]{\textcolor[rgb]{0.81,0.36,0.00}{\textbf{#1}}}
\newcommand{\OtherTok}[1]{\textcolor[rgb]{0.56,0.35,0.01}{#1}}
\newcommand{\PreprocessorTok}[1]{\textcolor[rgb]{0.56,0.35,0.01}{\textit{#1}}}
\newcommand{\RegionMarkerTok}[1]{#1}
\newcommand{\SpecialCharTok}[1]{\textcolor[rgb]{0.00,0.00,0.00}{#1}}
\newcommand{\SpecialStringTok}[1]{\textcolor[rgb]{0.31,0.60,0.02}{#1}}
\newcommand{\StringTok}[1]{\textcolor[rgb]{0.31,0.60,0.02}{#1}}
\newcommand{\VariableTok}[1]{\textcolor[rgb]{0.00,0.00,0.00}{#1}}
\newcommand{\VerbatimStringTok}[1]{\textcolor[rgb]{0.31,0.60,0.02}{#1}}
\newcommand{\WarningTok}[1]{\textcolor[rgb]{0.56,0.35,0.01}{\textbf{\textit{#1}}}}
\usepackage{graphicx,grffile}
\makeatletter
\def\maxwidth{\ifdim\Gin@nat@width>\linewidth\linewidth\else\Gin@nat@width\fi}
\def\maxheight{\ifdim\Gin@nat@height>\textheight\textheight\else\Gin@nat@height\fi}
\makeatother
% Scale images if necessary, so that they will not overflow the page
% margins by default, and it is still possible to overwrite the defaults
% using explicit options in \includegraphics[width, height, ...]{}
\setkeys{Gin}{width=\maxwidth,height=\maxheight,keepaspectratio}
\setlength{\emergencystretch}{3em}  % prevent overfull lines
\providecommand{\tightlist}{%
  \setlength{\itemsep}{0pt}\setlength{\parskip}{0pt}}
\setcounter{secnumdepth}{0}
% Redefines (sub)paragraphs to behave more like sections
\ifx\paragraph\undefined\else
\let\oldparagraph\paragraph
\renewcommand{\paragraph}[1]{\oldparagraph{#1}\mbox{}}
\fi
\ifx\subparagraph\undefined\else
\let\oldsubparagraph\subparagraph
\renewcommand{\subparagraph}[1]{\oldsubparagraph{#1}\mbox{}}
\fi

% set default figure placement to htbp
\makeatletter
\def\fps@figure{htbp}
\makeatother


\title{Chapter 2 - Exercises}
\author{Héctor Lira Talancón}
\date{4/4/2020}

\begin{document}
\maketitle

\hypertarget{chapter-summary}{%
\section{Chapter Summary}\label{chapter-summary}}

Descriptive methods in survival data:

\hypertarget{kaplan-meier-curves}{%
\subsection{1. Kaplan-Meier Curves}\label{kaplan-meier-curves}}

\[\hat{S}(t)=\prod\limits_{t_{(1)} \leq t} \frac{n_i - d_i}{n_i}\]

with \(\hat{S}(t)=1\) if \(t < t_{(1)}\)

\hypertarget{life-table-estimator-for-grouped-data}{%
\subsection{2. Life-Table Estimator (for
grouped-data)}\label{life-table-estimator-for-grouped-data}}

\begin{itemize}
\tightlist
\item
  Group data as in a histogram-like fashion
\item
  Use Kaplan-Meier curves as before
\item
  Modify the risk set size by (\# censored / 2)
\end{itemize}

\[\big(n - (c/2) - d\big) / \big(n - (c/2)\big)\]

Things to consider:

\begin{itemize}
\tightlist
\item
  The number of subjects being followed affects the magnitude of the
  descends.
\item
  Point estimates, confidence intervals, and estimates of key quantiles
  are needed to asses the precision of our survival curves.
\end{itemize}

\hypertarget{variance-of-the-kaplan-meier-curve-delta-method}{%
\subsection{3. Variance of the Kaplan-Meier Curve (delta
method):}\label{variance-of-the-kaplan-meier-curve-delta-method}}

\begin{itemize}
\tightlist
\item
  For Kaplan-Meier, we use the delta method (uses first-order Taylor
  series expansions) to estimate the variance (it is easier to estimate
  the log-variance because it is easier to estimate the variance of a
  sum than of a product).
\item
  The log of the KM estimator is:
\end{itemize}

\[\ln{\hat{S}(t)} = \sum\limits_{t_{(1)} \leq t}\ln \frac{n_i - d_i}{n_i} = \sum\limits_{t_{(1)} \leq t}\ln \hat{p}_i\]

\begin{itemize}
\tightlist
\item
  If we consider the observations at time \(t_i\) to be independent
  Bernoulli random variables (with constant probability), then, each
  \(\hat{p}_i\) is an estimator of this probability. The variance of
  this estimated probability is \(p_i(1-p_i) / n_i\).
\item
  Using the delta method (calculate the variance of \(\exp{X}\) with
  \(X=\ln{\hat{S}(t)}\)), we retrieve the estimator of
  \(\text{Var}[\hat{S}(t)]\):
\end{itemize}

\[\hat{\text{Var}}[\hat{S}(t)] = \big[\hat{S}(t)\big]^2 \sum\limits_{t_{(1)}\leq t} \frac{d_i}{n_i(n_i-d_1)}\]

\hypertarget{confidence-interval-estimates}{%
\subsection{4. Confidence Interval
Estimates}\label{confidence-interval-estimates}}

\begin{itemize}
\tightlist
\item
  It has been proven that the Kaplan-Meier estimates and functions of it
  are asymptotically normal.
\item
  However, using the normal distribution quantiles may lead to negative
  values.
\item
  In addition, the assumption of normality may not hold for small
  samples.
\item
  Kalbfleisch and Prentice suggest that confidence interval estimation
  should be based on the \emph{log-log} survival function,
  \(\ln\{-\ln \hat{S}(t)\}\), which has a range of
  \((-\infty, \infty)\).
\item
  The estimator of the variance is obtained using one more time the
  delta method:
\end{itemize}

\[\hat{\text{Var}}\Big[\ln\{-\ln \hat{S}(t)\}\Big] = \frac{1}{\big[\ln \hat{S}(t)\big]^2} \sum\limits_{t_{(1)}\leq t} \frac{d_i}{n_i(n_i-d_1)}\]
- The endpoints of a \(100(1-\alpha)\%\) confidence interval is given
by:

\[\ln\{-\ln \hat{S}(t)\} \pm z_{1-\alpha/2} \hat{\text{SE}}\big[\ln\{-\ln \hat{S}(t)\}\big]\]
where \(z_{1-\alpha/2}\) is the \(\alpha/2\) percentile of a standard
normal distribution and \(\hat{\text{SE}}(\cdot)\) is the estimated
standard error of \((\cdot)\), in this case, the estimated variance of
the \emph{log-log} survival function.

\begin{itemize}
\tightlist
\item
  If we let \(\hat{c}_l\) and \(\hat{c}_u\) denote the lower and upper
  limits of the interval mentioned above, it follows that the confidence
  interval for the survival function is given by:
\end{itemize}

\[\big(\exp\big[-\exp(\hat{c}_u)\big], \exp\big[-\exp(\hat{c}_l)\big]\big)\]

Note how the endpoints are inverted.

\begin{itemize}
\tightlist
\item
  It has been observed that this estimation performs well with sample
  sizes as small as 25 observations and with up to \(50\%\)
  right-censored observations.
\item
  Other confidence interval estimations include different
  transformations: (already studied) log transformation,
  \(\ln\hat{S}(t)\), the logit transformation,
  \(\ln{\hat{S}(t) / \big[1-\hat{S}(t)\big]}\), and the arcsine
  transformation, \(\arcsin \sqrt{\hat{S}(t)}\).
\item
  There is little practical differences from using a different
  transformation, except that the the arcsine transformation gets wider
  faster than the other ones for small size samples.
\end{itemize}

\hypertarget{confidence-interval-bands}{%
\subsubsection{Confidence Interval
Bands}\label{confidence-interval-bands}}

\begin{itemize}
\tightlist
\item
  Using the confidence interval estimations we can compute confidence
  interval bands.
\item
  However, the probability that all of the confidence intervals contain
  their respective parameter is much less than \((1-\alpha/2)\).
\item
  Hall and Wellner suggest the following formula to compute interval
  bands for the interval \((0, t_{(m)})\) where \(t_{(m)}\) is the
  largest non-censored value of time:
\end{itemize}

\[\ln\big[-\ln\hat{S}(t)\big] \pm H_{\hat{a},\alpha} \frac{1+n\hat{\sigma}^2(t)}{\sqrt{n}|\ln\hat{S}(t)|}\]
where

\[\hat{a} = n{\sigma}^2(t_{(m)}) / \big[1+n{\sigma}^2(t)\big]\],

\[\hat{\sigma}^2(t) = \sum\limits_{t_{(1)} \leq t} \frac{d_i}{n_i(n_i-d_1)}\]

and \(H_{\hat{a},\alpha}\) is a percentile of the following table:

\begin{Shaded}
\begin{Highlighting}[]
\NormalTok{hw_table <-}\StringTok{ }\KeywordTok{data.frame}\NormalTok{(}\StringTok{"1-a"}\NormalTok{=}\KeywordTok{c}\NormalTok{(}\FloatTok{0.9}\NormalTok{, }\FloatTok{0.95}\NormalTok{, }\FloatTok{0.99}\NormalTok{),}
                       \StringTok{"0.1"}\NormalTok{=}\KeywordTok{c}\NormalTok{(}\FloatTok{0.599}\NormalTok{, }\FloatTok{0.682}\NormalTok{, }\FloatTok{0.851}\NormalTok{))}

\NormalTok{hw_table}
\end{Highlighting}
\end{Shaded}

\begin{verbatim}
##   X1.a  X0.1
## 1 0.90 0.599
## 2 0.95 0.682
## 3 0.99 0.851
\end{verbatim}

\begin{itemize}
\tightlist
\item
  Denote the lower and upper endpoints of this interval as \(\hat{b}_l\)
  and \(\hat{b}_u\), respectively. The endpoints of the survival
  function are:
\end{itemize}

\[\Big(\exp\big[-\exp(\hat{b}_u)\big], \exp\big[-\exp(\hat{b}_l)\big]\Big)\]

\begin{itemize}
\tightlist
\item
  With this confidence band estimate, we can say with a \(95\%\)
  probability that the band will include its parameter.
\end{itemize}

\end{document}
